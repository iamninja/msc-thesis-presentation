%!TEX root = main.tex

\section{o-Minimality}
\subsection{Introduction}


\begin{frame}[c]{o-Minimality}
	\begin{block}{Assumptions}
		\begin{itemize}
			\item $\mathcal{M}=(M,<,\ldots)$
			\item $<$ is dense, linear, without endpoints
			\item definability with parameters
		\end{itemize}
	\end{block}

	\begin{definition}
		The structure $\mathcal{M}$ is called \em o-minimal \em if every definable subset of $M$ is a finite union of singletons and open intervals with endpoints in $M_{\infty}:=M\cup\{-\infty,+\infty\}$.\\
		A theory $T$ is called \em o-minimal \em if every model $\mathcal{M}$ of $T$ is o-minimal.
	\end{definition}
\end{frame}

\begin{frame}[c]{o-Minimality}
	The class of o-minimal structures is,
		\begin{itemize}
			\item closed under reducts (if $<$ still remains in the language)
			\item closed under expansions by constants
		\end{itemize}

	\begin{exampleblock}{Some o-minimal structures}
		\begin{itemize}
			\item $(\mathbb{Q},<)$
			\item $(\mathbb{Q},<,+)$
			\item $\mathcal{R}=(\mathbb{R},<,+,-,\cdot,0,1)$
		\end{itemize}
	\end{exampleblock}

	\begin{block}{$(\mathbb{Q},<,+,\cdot,0,1)$ is not o-minimal}
		The infinite discrete set of perfect squares is definable.
	\end{block}
\end{frame}

\subsection{Monotonicity and Finiteness}

\begin{frame}[c]{Monotonicity and Finiteness}
	\begin{block}{Monotonicity Theorem}
		Let $\mathcal{M}$ be an o-minimal structure and $f:(a,b) \to M$ be a definable function with domain $(a,b)$(possibly $a=- \infty$ or $b=+ \infty$).
		Then, there are points $a=a_0<a_1< \cdots <a_{k+1}$ s.t. for each $j=0,\ldots,k$, $f|_{(a_j,a_{j+1})}$ is either,
		\begin{itemize}
			\item constant, or
			\item a strictly monotonic and continuous bijection to an interval.
		\end{itemize}
 	\end{block}
 	% The proof of this utilizes the following lemma.
 	% \begin{block}{Lemma}
 	% 	Let $f$ be a function as described in the Monotonicity Theorem.
 	% 	Then for any definable infinite interval $I \subseteq (a,b)$, there is an infinite $I^* \subseteq I$ on which $f$ is either constant or strictly monotone.
 	% \end{block}

 	\begin{beamerboxesrounded}[shadow=true]{Finiteness Lemma}
 		Let $A \subseteq M^2$ be definable and suppose that for each $x \in M$ the fiber $A_x := \{ y \in M|(x,y) \in A \}$ is finite.
		Then there is $N < \omega$ s.t. $|A_x| \leq N$ for all $x \in M$. 
 	\end{beamerboxesrounded}
\end{frame}

% \begin{frame}[t]\frametitle{Proof of Monotonicity Theorem}
%     Conside the formula $\theta(x)$
%     \begin{beamerboxesrounded}[shadow=true]{}
%     	``On an interval of which $x$ is the left endpoint, $f$ is strictly monotone or constant, and there is no interval extending this interval on the left on which $f$ is strictly monotone or constant.''
%     \end{beamerboxesrounded}
% \end{frame}

\begin{frame}[c]\frametitle{Applications}	
    
	\begin{block}{Corollary}
		Let $f:(a,b) \to M$ be definale and continuous. Then $f$ takes a maximum and minimum value on $[a,b]$.
	\end{block}

	\begin{block}{Exchange Principle}
		Let $\mathcal{M}$ be o-minimal.
		Let $b,c,a_1,\ldots,a_n \in \mathcal{M}$.
		If $b$ is definable over $c,a_1,\ldots,a_n$, and $b$ is not definable over $a_1,\ldots,a_n$, then $c$ is definable over $b,a_1,\ldots,a_n$.
	\end{block}

\end{frame}

\subsection{Cell Decomposition}

\begin{frame}[c]\frametitle{Motivation}
	
	\begin{beamerboxesrounded}[shadow=true]{Question}
		What happens with the definable subsets of $M^n$?
	\end{beamerboxesrounded}
    
	\begin{beamerboxesrounded}[shadow=true]{Notation}
		Given definable $X \subseteq M^n$
		\begin{itemize}
			\item $C(X):= \{ f:X \to M|f \text{ is definable and continuous}\}$
			\item $C_\infty(X):=C(X) \cup \{ -\infty, +\infty \}$
			\item for $f,g \in C_\infty(X)$ s.t. $(\forall \bar{x} \in X)(f(\bar{x})<g(\bar{x}))$ then 
			$$(f,g)_X := \{  (\bar{x},y)\in X \times M|f(\bar{x})<y<g(\bar{x})\}$$
		\end{itemize}
	\end{beamerboxesrounded}

\end{frame}

\begin{frame}[c]\frametitle{Cells}

		Let $(i_1,\ldots,i_n)$ be a binary sequence. Then a $(i_1,\ldots,i_n)$-\em cell \em is a definable subset of $M^n$ defined as follows,
		\begin{columns}

			\begin{column}{7cm}
					\begin{itemize}
						\item a $(0)$-cell is a singleton
						\item a $(1)$-cell is an open interval
					\end{itemize}
					Suppose that we have defined $(i_1,\ldots,i_{n-1})$-cells, then
					\begin{itemize}
						\item<2-> a $(i_1,\ldots,i_{n-1},0)$-cell is the graph of some $f \in C(X)$,
						where $X$ is a  $(i_1,\ldots,i_{n-1})$-cell
						\item<3-> a $(i_1,\ldots,i_{n-1},1)$-cell is the set $(f,g)_X$ where $X$ is a
						$(i_1,\ldots,i_{n-1})$-cell and $f,g \in C_\infty(X)$
					\end{itemize}

			\end{column}

			\begin{column}{4cm}

				\temporal<2>{}{\includegraphics[height=5cm,width=4cm]{celliii0-.pdf}}{\includegraphics[height=5cm,width=4cm]{cell-.pdf}}


			\end{column}


		\end{columns}
\end{frame}

\begin{frame}[c]\frametitle{Decompositions}
    
	A \em decomposition \em of $M^n$ is a partition of $M^n$ into finitely many cells. It is defined recursively.

	\begin{itemize}
		\item A decomposition of $M$ is a collection,
		$$\{  (-\infty,a_1),(a_1,a_2),\ldots,(a_k,+\infty),\{a_1\},\ldots,\{a_k\} \}$$
		where $a_1<\ldots<a_k$ are points in $M$;
		\item a decomposition of $M^{n+1}$ is a finite partition of $M^{n+1}$ into cells $C$ s.t. the set of projections $\pi(C)$ is a decomposition of $M^n$.
	\end{itemize}

	A decomposition $\mathcal{D}$ of $M^n$ is said to \em partition \em a set $S \subseteq M^n$ if $S$ is a union of cells in $\mathcal{D}$.

\end{frame}

\begin{frame}[c]\frametitle{Cell Decomposition Theorem}
    
	\begin{beamerboxesrounded}[shadow=true]{Uniform Finiteness Property}
		Let $Y \subseteq M^{n+1}$ be definable and also for every $\bar{x} \in M^n$, 
		$$Y_{\bar{x}}:= \{  r \in M | (\bar{x},r) \in Y \}$$ is finite.
		Then there exists $N<\omega$ s.t. $Y_{\bar{x}} \leq N$ for all $\bar{x} \in M^n$.
	\end{beamerboxesrounded}

	\begin{beamerboxesrounded}[shadow=true]{Cell Decomposition Theorem}
		\begin{enumerate}
			\item Given any definable sets $A_1,\ldots,A_k\subseteq M^n$ there is a decomposition of $M^n$ partitioning each of $A_1,\ldots,A_k$.
			\item For each definable function $f:A \to M$, $A \subseteq M^n$, there is a decomposition $\mathcal{D}$ of $M^n$ partitioning $A$ s.t. the restriction $f|B:B\to M$ to each cell $B \in \mathcal{D}$ with $B\subseteq A$ is continuous.
		\end{enumerate}
	\end{beamerboxesrounded}

\end{frame}

\begin{frame}[c]\frametitle{Applications}
    
	\begin{beamerboxesrounded}[shadow=true]{Theorem}
		If $\mathcal{M}$ is an o-minimal structure, the $\text{Th}(\mathcal{M})$ is an o-minimal theory.
	\end{beamerboxesrounded}
	Proof...

\end{frame}