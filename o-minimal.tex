%!TEX root = main.tex

\section{o-Minimality}
\subsection{Introduction}


\begin{frame}{o-Minimality}
	\begin{block}{Assumptions}
		\begin{itemize}
			\item $\mathcal{M}=(M,<,\ldots)$
			\item $<$ is dense, linear, without endpoints
			\item definability with parameters
		\end{itemize}
	\end{block}

	\begin{definition}
		The structure $\mathcal{M}$ is called \em o-minimal \em if every definable subset of $M$ is a finite union of singletons and open intervals with endpoints in $M_{\infty}:=M\cup\{-\infty,+\infty\}$.\\
		A theory $T$ is called \em o-minimal \em if every model $\mathcal{M}$ of $T$ is o-minimal.
	\end{definition}
\end{frame}

\begin{frame}{o-Minimality}
	The class of o-minimal structures is,
		\begin{itemize}
			\item closed under reducts (if $<$ still remains in the language)
			\item closed under expansions by constants
		\end{itemize}

	\begin{exampleblock}{Some o-minimal structures}
		\begin{itemize}
			\item $(\mathbb{Q},<)$
			\item $(\mathbb{Q},<,+)$
			\item $\mathcal{R}=(\mathbb{R},<,+,-,\cdot,0,1)$
		\end{itemize}
	\end{exampleblock}

	\begin{block}{$(\mathbb{Q},<,+,\cdot,0,1)$ is not o-minimal}
		The infinite discrete set of perfect squares is definable.
	\end{block}
\end{frame}

\subsection{Monotonicity and Finiteness}

\begin{frame}{Monotonicity and Finiteness}
	\begin{block}{Monotonicity Theorem}
		Let $\mathcal{M}$ be an o-minimal structure and $f:(a,b) \to M$ be a definable function with domain $(a,b)$(possibly $a=- \infty$ or $b=+ \infty$).
		Then, there are points $a=a_0<a_1< \cdots <a_{k+1}$ s.t. for each $j=0,\ldots,k$, $f|_{(a_j,a_{j+1})}$ is either,
		\begin{itemize}
			\item constant, or
			\item a strictly monotonic and continuous bijection to an interval.
		\end{itemize}
 	\end{block}
 	% The proof of this utilizes the following lemma.
 	% \begin{block}{Lemma}
 	% 	Let $f$ be a function as described in the Monotonicity Theorem.
 	% 	Then for any definable infinite interval $I \subseteq (a,b)$, there is an infinite $I^* \subseteq I$ on which $f$ is either constant or strictly monotone.
 	% \end{block}

 	\begin{beamerboxesrounded}[shadow=true]{Finiteness Lemma}
 		Let $A \subseteq M^2$ be definable and suppose that for each $x \in M$ the fiber $A_x := \{ y \in M|(x,y) \in A \}$ is finite.
		Then there is $N < \omega$ s.t. $|A_x| \leq N$ for all $x \in M$. 
 	\end{beamerboxesrounded}
\end{frame}

% \begin{frame}[t]\frametitle{Proof of Monotonicity Theorem}
%     Conside the formula $\theta(x)$
%     \begin{beamerboxesrounded}[shadow=true]{}
%     	``On an interval of which $x$ is the left endpoint, $f$ is strictly monotone or constant, and there is no interval extending this interval on the left on which $f$ is strictly monotone or constant.''
%     \end{beamerboxesrounded}
% \end{frame}

\begin{frame}[t]\frametitle{Applications}	
    
	\begin{block}{Corollary}
		Let $f:(a,b) \to M$ be definale and continuous. Then $f$ takes a maximum and minimum value on $[a,b]$.
	\end{block}

	\begin{block}{Exchange Principle}
		Let $\mathcal{M}$ be o-minimal.
		Let $b,c,a_1,\ldots,a_n \in \mathcal{M}$.
		If $b$ is definable over $c,a_1,\ldots,a_n$, and $b$ is not definable over $a_1,\ldots,a_n$, then $c$ is definable over $b,a_1,\ldots,a_n$.
	\end{block}

\end{frame}