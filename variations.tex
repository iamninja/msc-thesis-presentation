%!TEX root = main.tex

\section{Variations}
\subsection{Weak o-minimality}

\begin{frame}[c]\frametitle{Weakly o-minimal structures}
	
	Let $\mathcal{M} = (M,<, \dots )$ be a linearly ordered structure.
    
	\begin{beamerboxesrounded}[shadow=true]{Definition}
		A set $C \subseteq M$ is called \em convex \em, if for any $a,b \in C$ with $a<b$, and $c\in M$ such that $a<c<b$, then $c \in C$. 
	\end{beamerboxesrounded}

	\begin{beamerboxesrounded}[shadow=true]{Definition}
		A structure $\mathcal{M}$ will be called \em weakly o-minimal \em, if the definable subsets of $\mathcal{M}$ are finite unions of convex sets in $(M,<)$.\\
		We say that a complete theory $T$ is \em weakly o-minimal \em if every model of $T$ is weakly o-minimal. 
	\end{beamerboxesrounded}

\end{frame}